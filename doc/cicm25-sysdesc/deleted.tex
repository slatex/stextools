


% In this paper, we address tools that help attack the coverage problem in symbolic/semantic
% approaches in practice. The producer task we use as a case study is that of adaptive
% learning assistants for STEM subjects, concretely the \ALeA system \cite{BerBetChu:lssmkm23}.
% \ALeA uses \sTeX \cite{MueKo:sdstex22,sTeX:github:on} -- a
% variant of {\LaTeX} that allows to embed semantic annotations -- for knowledge
% representation, and generates learner-adaptive learning objects instrumented with
% learning-support interactions from that. The \textbf{\sTeX/\ALeA content commons}
% \ednote{MK@MK: continue to describe the size and distribution, and
%   authorship}\ednote{explain the \textbf{domain model} (the SMGloM) and the
%   \textbf{formulation model} (cf. \cite{BerBetChu:lssmkm23}) in the commons and give their
%   relative sizes.}

% In this paper will focus on adaptive learning assistants for STEM
% subjects in tertiary education,
% concretely the \ALeA system \cite{BerBetChu:lssmkm23}.


% Flexiformal\ednote{
%     \emph{semantic authoring} already used in context of semantic web (RDFa).
%     Tools exist and it doesn't have exactly the same challenges as we face,
%     though there is some overlap. We have a module system and need
%     more dependency management, something else?
%     This has to be described somewhere.
% 
%     Related work:
%     ``User Interfaces for Semantic Authoring of Textual Content:
%     A Systematic Literature Review.''
% }
% authoring lies between traditional informal authoring
% and formal authoring (like in proof assistants)
% where documents are designed for computer processing.
% During informal authoring, the focus is on presenting content to a human reader,
% which results in documents that are largely inscrutable by computers.
