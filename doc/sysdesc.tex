\documentclass{llncs}
\usepackage[show]{ed}
\usepackage{cleveref}
\usepackage{wrapfig}
\usepackage{xspace}
\usepackage{stex-logo}

\usepackage[style=alphabetic,hyperref=auto,defernumbers=true,backend=bibtex,firstinits=true,maxbibnames=9,maxcitenames=3,isbn=false]{biblatex}
\addbibresource{kwarcpubs.bib}
\addbibresource{extpubs.bib}
\addbibresource{kwarccrossrefs.bib}
\addbibresource{extcrossrefs.bib}

% I do not want to annotate just yet.

\newcommand\ALeA{\textsf{ALeA}\xspace}

\title{Bulk Semantic Annotation with an Partially Known Knowledge Base}
\author{Michael Kohlhase, Jan Frederik Schaefer}
\institute{Computer Science, FAU Erlangen N\"urnberg, Germany}
\begin{document}
\maketitle
\begin{abstract}
tbw
\end{abstract}

\section{Introduction}
Arguably, dealing with large document collections is one of the key factors in the
knowledge-driven society and economy. There are currently two main contenders for machine
support in this area. Symbolic/logic-based technologies and subsymbolic,
machine-learning-based AI, e.g. via LLMs or chatbots; they have complementary strengths
and challenges: Symbolic technologies offer precision and explainability out of the box,
but face scalability challenges, because the prerequisite background knowledge has to be
(manually) formalized. ML-based approaches can be trained on all data of the Internet, but
face challenges in precision and explanations. Aarne Ranta matches these profiles to the
notion of \textbf{producer tasks} -- i.e. tasks where precision is key, but limited
coverage ($10^{3\pm1}$ concepts) is ok; e.g. for multi-language/variant manuals for very
expensive tool machines -- and \textbf{consumer tasks}, where coverage is key and
precision secondary; e.g. for consumer-grade machine translation like \textsf{Google
  translate} in \cite{Ranta:atcp17}.

In this paper, we address tools that help attack the coverage problem in symbolic/semantic
approaches in practice. The producer task we use as a case study is that of adaptive
learning assistants for MINT subject, concretely the \ALeA system
\cite{BerBetChu:lssmkm23}, which uses \sTeX \cite{MueKo:sdstex22,URL:sTeX:github} -- a
variant of {\LaTeX} that allows to embed semantic annotations -- for a knowledge
representation and generates learner-adaptive learning objects instrumented with
learning-support interactions from that. The \ALeA ecosystem 

\section{Conclusion}
\printbibliography
\end{document}

%%% Local Variables:
%%% mode: latex
%%% TeX-master: t
%%% End:
